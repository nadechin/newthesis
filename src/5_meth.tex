\chapter{本研究の手法}
\label{meth}

本章では,~\ref{appr:Hypothesis}節で述べた仮説を検証するために,本研究で行なった手法について概説する.

\section{問題解決の為のアプローチ}
\label{meth:appr}
 ~\ref{appr:YotenForProblem}で述べた問題解決のための2つの要件を,本研究の手法として提案する.

 \subsection{コマンドの追加実装}
実際のShellに実装されているコマンドで,SSHの低対話型Honeypotに実装されていないコマンドを実装する.これによってコマンドの追加実装を行なった低対話型Honeypotに侵入した侵入者は,実際のShellと同じような挙動をする低対話型HoneypotをHoneypotであると検知できなくなる.

 \subsection{既実装コマンドの修正}
 SSHの低対話型Honeypotに特有の異常な挙動をする既実装コマンドを修正する.これによって既実装コマンドの修正を行なった低対話型Honeypotに侵入した侵入者は,実際のShellと同じような挙動をする低対話型HoneypotをHoneypotであると検知できなくなる.



%%% Local Variables:
%%% mode: japanese-latex
%%% TeX-master: "../bthesis"
%%% End:
