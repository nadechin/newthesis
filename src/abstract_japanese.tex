卒業論文要旨 - 2018年度 (平成30年度)
\begin{center}
\begin{large}
\begin{tabular}{|M{0.97\linewidth}|}
    \hline
   低対話型Honeypotのコマンド拡張による\\収集ログの変化の計測\\
    \hline
\end{tabular}
\end{large}
\end{center}

~ \\

PCの普及やIoTデバイスのシステム高度化により,高度な処理系を組むことが可能になった.これによりデバイス上にLinux系などのOSが搭載された機器が広く人々に使われるようになった.また,Linux系OSにリモートログインする手法としてSSHがある.これを用いて不正に侵入する攻撃が行われている.
侵入された際に侵入者がどのような挙動をしているのかを知る手段として,Honeypotがある.HoneypotはSSHで侵入しやすいような環境を作ることで,侵入者にログイン試行に成功したと検知させ,その際に実行したコマンドのログを収集するものである.また現在ではShellの挙動をエミュレートしたHoneypotが広く使用されており,このHoneypotは実行できるコマンドが少ない実装になっている.そのためHoneypotへの侵入者に侵入先がHoneypotであると検知されてしまう.そこで事前実験ではHoneypotのコマンドを拡張し,拡張をしていないHoneypotとコマンドの拡張をしたHoneypotで侵入ログを収集した.収集したログを確率的な算出方法を使用することで比較した結果,より多くの侵入者のコマンド実行ログのパターンを取得できることを示した.本研究ではコマンドを拡張したHoneypotの侵入ログがどれほど実際のOSに不正なSSHの侵入をされた際の侵入ログが近似したのかを検証した.評価として,拡張をしていないHoneypotとコマンドの拡張をしたHoneypotと,さらに実際のOSを使用したHoneypotで侵入ログを収集し,比較を行なった.この3つの侵入ログを自然言語処理の意味解析を用い,一セッションにおけるコマンドログの意味をベクトル空間上に表現することで,拡張したHoneypotで収集した侵入ログが,拡張をしていないHoneypotで収集した侵入ログよりも,一般的なUNIXユーザーの実行するコマンドログから離れることを示した.

~ \\
キーワード:\\
\underline{1. 自然言語処理},
\underline{2. 意味解析},
\underline{3. Honeypot},
\underline{4. SSH}
\begin{flushright}
慶應義塾大学 総合政策学部\\
菅藤 佑太
\end{flushright}
