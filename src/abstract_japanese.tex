卒業論文要旨 - 2018年度 (平成30年度)
\begin{center}
\begin{large}
\begin{tabular}{|M{0.97\linewidth}|}
    \hline
   低対話型Honeypotのコマンド拡張による\\高対話型Honeypotへの近似\\
    \hline
\end{tabular}
\end{large}
\end{center}

~ \\

PCの普及やIoTデバイスのシステム高度化により,高度な処理系を組むことが可能になり,デバイス上にLinux系などのOSが動く機器が広く人々に使われるようになった.そうした中でSSHの不正な侵入によって自分の機器が踏み台にされ,自身の機器から第三者のネットワーク機器に攻撃されていたり,ウイルスやバックドアが設置されて自身の機器が不正にウイルスが実行されるような環境を作られる問題が発生している.
侵入された際に侵入者がどのような挙動をしているのかを知る手段として,Honeypotというものがある.これはSHELLの擬似的な挙動をアプリケーション上で実現し,敢えてSSHで侵入しやすいような環境を作ることで,侵入者にログイン試行に成功したと検知させ,その際に実行したコマンドを観測する.本研究では,Honeypotの機能を拡張することでより多くの,高精度な攻撃ログを取得し,これを評価した.


~ \\
キーワード:\\
\underline{1. SSH},
\underline{2. Honeypot},
\underline{3. 機械学習},
\underline{4. 自然言語処理}
\begin{flushright}
慶應義塾大学 総合政策学部\\
菅藤 佑太
\end{flushright}
