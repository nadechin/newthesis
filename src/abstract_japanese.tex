卒業論文要旨 - 2018年度 (平成30年度)
\begin{center}
\begin{large}
\begin{tabular}{|M{0.97\linewidth}|}
    \hline
   低対話型Honeypotのコマンド拡張による\\高対話型Honeypotへの近似\\
    \hline
\end{tabular}
\end{large}
\end{center}

~ \\

PCの普及やIoTデバイスのシステム高度化により,高度な処理系を組むことが可能になった.これによりデバイス上にLinux系などのOSが動く機器が広く人々に使われるようになった.そうした中でSSHの不正な侵入によって自分の機器が踏み台にされ,自身の機器から第三者のネットワーク機器に攻撃されていたり,ウイルスやバックドアが設置されて自身の機器が不正にウイルスが実行されるような環境を作られる問題が発生している.
侵入された際に侵入者がどのような挙動をしているのかを知る手段として,Honeypotがある.敢えてSSHで侵入しやすいような環境を作ることで,侵入者にログイン試行に成功したと検知させ,その際に実行したコマンドのログを収集するものである.また現在ではShellの挙動をエミュレートしたHoneypotが広く使用されており,このHoneypotは実行できるコマンドが少ない実装になっている.そのためHoneypotへの侵入者に侵入先がHoneypotであると検知されてしまう.そこで事前実験ではHoneypotのコマンドを拡張し,拡張をしていないHoneypotとコマンドの拡張をしたHoneypotで侵入ログを収集した.収集したログを確率的な算出方法を使用することで比較した結果,より多くの侵入者のコマンド実行ログのパターンを取得できることを示した.本研究ではコマンドを拡張したHoneypotの侵入ログがどれほど実際のOSに不正なSSHの侵入をされた際の侵入ログに近似したのかをを示すために,拡張をしていないHoneypotとコマンドの拡張をしたHoneypotと,さらに実際のOSを使用したHoneypotで侵入ログを収集した.この3つの侵入ログを自然言語処理の意味解析を用い,コマンドの一つ一つの意味をベクトル表現することで,拡張したHoneypotで収集した侵入ログが,拡張をしていないHoneypotで収集した侵入ログよりも,実際のOSを使用したHoneypotで侵入ログに意味的に近くなることを評価した.


~ \\
キーワード:\\
\underline{1. SSH},
\underline{2. Honeypot},
\underline{3. 機械学習},
\underline{4. 自然言語処理}
\begin{flushright}
慶應義塾大学 総合政策学部\\
菅藤 佑太
\end{flushright}
