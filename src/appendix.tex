\appendix
\chapter{付録}
\label{appd}

\section{実装コマンド}

\subsection{純正のHoneypotで未実装のコマンド種類}
\label{appd:kindofcommand}

\begin{center}
\large{\textbf{実装コマンド一覧}}
\end{center}

BusyBoxに含まれるコマンドとCowrieの実装コマンドの違い
 
\begin{longtable}{p{60mm}p{60mm}}
 \caption{実装コマンド一覧}
 \label{table:command} \\
 %------ 最初のページの表の最上部 ----
 \hline
 Busyboxに含まれるコマンド & Cowrieに実装されているか \\ \hline\hline
 \endfirsthead
 %------ 2ページ以降の表の最上部 ----
 \multicolumn{2}{r}{前ページからの続き} \\ \hline
 Busyboxに含まれるコマンド & Cowrieに実装されているか \\ \hline\hline
 \endhead
 %----- ページの表の最下部 --------
 \hline
 \multicolumn{2}{r}{次ページに続く} \\
 \endfoot
 %----- 最終ページの表の最下部 --------
 \hline
 \multicolumn{2}{r}{以上} \\
 \endlastfoot
acpid & ○ \\ \hline
adjtimex & ○ \\ \hline
ar & ○ \\ \hline
arp & ○ \\ \hline
arping & ○ \\ \hline
ash & ○ \\ \hline
awk & × \\ \hline
basename & ○ \\ \hline
blockdev & ○ \\ \hline
brctl & × \\ \hline
bunzip2 & × \\ \hline
bzcat & × \\ \hline
bzip2 & × \\ \hline
cal & × \\ \hline
cat & ○ \\ \hline
chgrp & × \\ \hline
chmod & × \\ \hline
chown & ○ \\ \hline
chpasswd & ○ \\ \hline
chroot & × \\ \hline
chvt & × \\ \hline
clear & × \\ \hline
cmp & × \\ \hline
cp & ○ \\ \hline
cpio & ○ \\ \hline
crond & ○ \\ \hline
crontab & × \\ \hline
cttyhack & × \\ \hline
cut & × \\ \hline
date & ○ \\ \hline
dc & × \\ \hline
dd & × \\ \hline
deallocvt & × \\ \hline
depmod & ○ \\ \hline
devmem & ○ \\ \hline
df & × \\ \hline
diff & × \\ \hline
dirname & × \\ \hline
dmesg & ○ \\ \hline
dnsdomainname & ○ \\ \hline
dos2unix & × \\ \hline
dpkg & × \\ \hline
dpkg-deb & ○ \\ \hline
du & × \\ \hline
dumpkmap & ○ \\ \hline
dumpleases & × \\ \hline
echo & × \\ \hline
ed & × \\ \hline
egrep & × \\ \hline
env & × \\ \hline
expand & ○ \\ \hline
expr & × \\ \hline
FALSE & ○ \\ \hline
fdisk & ○ \\ \hline
fgrep & × \\ \hline
find & ○ \\ \hline
fold & × \\ \hline
free & × \\ \hline
freeramdisk & × \\ \hline
fstrim & × \\ \hline
ftpget & ○ \\ \hline
ftpput & × \\ \hline
getopt & × \\ \hline
getty & × \\ \hline
grep & × \\ \hline
groups & ○ \\ \hline
gunzip & ○ \\ \hline
gzip & ○ \\ \hline
halt & × \\ \hline
head & ○ \\ \hline
hexdump & × \\ \hline
hostid & ○ \\ \hline
hostname & × \\ \hline
httpd & ○ \\ \hline
hwclock & × \\ \hline
id & × \\ \hline
ifconfig & × \\ \hline
ifdown & × \\ \hline
ifup & ○ \\ \hline
init & ○ \\ \hline
insmod & × \\ \hline
ionice & ○ \\ \hline
ip & × \\ \hline
ipcalc & ○ \\ \hline
kill & ○ \\ \hline
killall & ○ \\ \hline
klogd & × \\ \hline
last & × \\ \hline
less & × \\ \hline
ln & × \\ \hline
loadfont & × \\ \hline
loadkmap & ○ \\ \hline
logger & ○ \\ \hline
login & ○ \\ \hline
logname & ○ \\ \hline
logread & ○ \\ \hline
losetup & ○ \\ \hline
ls & ○ \\ \hline
lsmod & ○ \\ \hline
lzcat & ○ \\ \hline
lzma & ○ \\ \hline
lzop & ○ \\ \hline
lzopcat & ○ \\ \hline
md5sum & ○ \\ \hline
mdev & ○ \\ \hline
microcom & ○ \\ \hline
mkdir & ○ \\ \hline
mkfifo & ○ \\ \hline
mknod & ○ \\ \hline
mkswap & ○ \\ \hline
mktemp & ○ \\ \hline
modinfo & ○ \\ \hline
modprobe & ○ \\ \hline
more & ○ \\ \hline
mount & ○ \\ \hline
mt & ○ \\ \hline
mv & ○ \\ \hline
nameif & ○ \\ \hline
nc & ○ \\ \hline
netstat & ○ \\ \hline
nslookup & ○ \\ \hline
od & ○ \\ \hline
openvt & ○ \\ \hline
passwd & ○ \\ \hline
patch & ○ \\ \hline
pidof & ○ \\ \hline
ping & ○ \\ \hline
ping6 & ○ \\ \hline
pivot_root & ○ \\ \hline
poweroff & ○ \\ \hline
printf & ○ \\ \hline
ps & ○ \\ \hline
pwd & ○ \\ \hline
rdate & ○ \\ \hline
readlink & ○ \\ \hline
realpath & ○ \\ \hline
reboot & ○ \\ \hline
renice & ○ \\ \hline
reset & ○ \\ \hline
rev & ○ \\ \hline
rm & ○ \\ \hline
rmdir & ○ \\ \hline
rmmod & ○ \\ \hline
route & ○ \\ \hline
rpm & ○ \\ \hline
rpm2cpio & ○ \\ \hline
run-parts & ○ \\ \hline
sed & ○ \\ \hline
seq & ○ \\ \hline
setkeycodes & ○ \\ \hline
setsid & ○ \\ \hline
sh & ○ \\ \hline
sha1sum & ○ \\ \hline
sha256sum & ○ \\ \hline
sha512sum & ○ \\ \hline
sleep & ○ \\ \hline
sort & ○ \\ \hline
start-stop-daemon & ○ \\ \hline
stat & ○ \\ \hline
static-sh & ○ \\ \hline
strings & ○ \\ \hline
stty & ○ \\ \hline
su & ○ \\ \hline
sulogin & ○ \\ \hline
swapoff & ○ \\ \hline
swapon & ○ \\ \hline
switch_root & ○ \\ \hline
sync & ○ \\ \hline
sysctl & ○ \\ \hline
syslogd & ○ \\ \hline
tac & ○ \\ \hline
tail & ○ \\ \hline
tar & ○ \\ \hline
taskset & ○ \\ \hline
tee & ○ \\ \hline
telnet & ○ \\ \hline
telnetd & ○ \\ \hline
test & ○ \\ \hline
tftp & ○ \\ \hline
time & ○ \\ \hline
timeout & ○ \\ \hline
top & ○ \\ \hline
touch & ○ \\ \hline
tr & ○ \\ \hline
traceroute & ○ \\ \hline
traceroute6 & ○ \\ \hline
TRUE & ○ \\ \hline
tty & ○ \\ \hline
tunctl & ○ \\ \hline
udhcpc & ○ \\ \hline
udhcpd & ○ \\ \hline
umount & ○ \\ \hline
uname & ○ \\ \hline
uncompress & ○ \\ \hline
unexpand & ○ \\ \hline
uniq & ○ \\ \hline
unix2dos & ○ \\ \hline
unlzma & ○ \\ \hline
unlzop & ○ \\ \hline
unxz & ○ \\ \hline
unzip & ○ \\ \hline
uptime & ○ \\ \hline
usleep & ○ \\ \hline
uudecode & ○ \\ \hline
uuencode & ○ \\ \hline
vconfig & ○ \\ \hline
vi & ○ \\ \hline
watch & ○ \\ \hline
watchdog & ○ \\ \hline
wc & ○ \\ \hline
wget & ○ \\ \hline
which & ○ \\ \hline
who & ○ \\ \hline
whoami & ○ \\ \hline
xargs & ○ \\ \hline
xz & ○ \\ \hline
xzcat & ○ \\ \hline
yes & ○ \\ \hline
zcat & ○ \\ \hline
\end{longtable}

%\subsection{純正のHoneypotで未実装のコマンドの実装}
%\label{appd:implofcommand}

%\begin{mylisting}[label={lst:acpid},language=sh,caption=acpid]
%# coding: utf-8

%import subprocess

%cmd = "acpid"
%subprocess.call(cmd.split())
%\end{mylisting}

%\begin{mylisting}[label={lst:acpid},language=sh,caption=adjtimex]
%# coding: utf-8

%import subprocess

%cmd = "adjtimex"
%subprocess.call(cmd.split())

%\end{mylisting}


%\begin{mylisting}[label={lst:acpid},language=sh,caption=adjtimex -p]
%# coding: utf-8

%import subprocess

%cmd = "adjtimex -p"
%subprocess.call(cmd.split())

%\end{mylisting}


%\begin{mylisting}[label={lst:acpid},language=sh,caption=ar]
%# coding: utf-8

%import subprocess
%import sys

%cmd = "ar t"
%opt = sys.argv[1]
%cmdall = cmd + " " + opt
%subprocess.call(cmdall.split())

%\end{mylisting}
%\begin{mylisting}[label={lst:acpid},language=sh,caption=arp]
%# coding: utf-8

%import subprocess

%cmd = "arp"
%subprocess.call(cmd.split())

%\end{mylisting}

%\begin{mylisting}[label={lst:acpid},language=sh,caption=arping]
%# coding: utf-8

%import subprocess

%cmd = "arping"
%subprocess.call(cmd.split())

%\end{mylisting}


%\begin{mylisting}[label={lst:acpid},language=sh,caption=ash]
%# coding: utf-8

%string = '''-bash: ash: command not found'''

%print(string)

%\end{mylisting}

%%% Local Variables:
%%% mode: japanese-latex
%%% TeX-master: "./thesis"
%%% End:
