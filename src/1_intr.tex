\chapter{序論}
\label{intr}

本章では本研究の背景,課題及び手法を提示し,本研究の概要を示す.

\section{通信機器の普及}
\label{intr:back}
PCの普及やIoTデバイスのシステム高度化により,高度な処理系を組むことが可能になった.これによりデバイス上にLinux系などのOSが搭載された機器が広く人々に使われるようになった.また,Linux系OSにリモートログインする手法としてSSHがある.これを用いて不正に侵入する攻撃が行われている.

\section{honeypot}
\label{intr:honey}
 侵入された際に侵入者がどのような挙動をしているのかを知る手段として,Honeypotがある.これは実際のOSを用いたり,Shellの擬似的な挙動をアプリケーション上で実現し,敢えてSSHで侵入しやすいような環境を作ることで,侵入者にログイン試行に成功したと検知させ,その際に実行したコマンドのログを収集する.

\section{本研究の問題と仮説}
\label{intr:prob}
SSHのHoneypotは大きく二種類に分けることができ,一つは低対話型Honeypot,もう一つは高対話型Honeypotである.低対話型Honeypotは実際のShellの挙動をエミュレートしたアプリケーションである.高対話型Honeypotは実際の機器を設置し,その中に侵入させログを収集する.その設置時には他のホストに攻撃できないようにネットワークの設定や,rootの権限が取られないようにuser権限の設定を適切に行う.高対話型Honeypotは低対話型Honeypotと比較すると,本物のOSを用いており,Honeypotへの侵入者が実行できるコマンドが多く,挙動も本物のOSと差異が極めて小さく,侵入先がHoneypotであると極めて検知しにくいため,高精度な攻撃ログを取得することができる.しかし,Honeypotとして適切な設定を行なったOSが,OSの脆弱性を突かれることで,OSが踏み台にされ他のホストに攻撃をしたりウイルスに犯されてしまうなどの危険を孕んでいるため,設置コストが高い.そのため,普及率も非常に低い\cite{HighIntHoney}.一方で低対話型Honeypotはアプリケーションであるため,root権限を取られるような危険が極めて少なく,アプリケーション内での脆弱性に限った問題しか存在しない.そのため設置コストが低く,比較的誰でも安全に設置できるため,普及率が高い.しかし,あくまでエミュレーションを行なったアプリケーションであるため,実際のShellとは異なる挙動や,Honeypotに特有な挙動をしまうことがある.そのため設置したHoneypotに侵入した悪意のあるユーザーに侵入先がHoneypotであると検知されてしまう可能性がある.\\
 本研究では低対話型Honeypotに着目する.低対話型で実際の攻撃ログに近いログを収集するには,先述のHoneypotであることの検知を回避する必要がある.そこで本研究では,低対話型に実装されているコマンドの出力を,実際のShellに近似することで検知を回避できるのではないかと考えた.

\section{予備実験}
\label{intr:prex}
SSHの低対話型Honeypotに実装されていないコマンドで悪意のある侵入者が使うコマンドを実装することで拡張を行なった上で,本研究の予備実験では,コマンドの追加実装を行なった低対話型Honeypotと,素の低対話型Honeypotでそれぞれ収集したコマンドログの比較を行なった.追加実装を施したSSHの低対話型Honeypotの方がコマンドパターンとして多く収集できることを示した.

\section{提案手法の実装}
\label{intr:impl}
先述のHoneypotであることの検知を回避するために,本研究では低対話型Honeypotを実際のShellの挙動に近似するために,2つの手法を行なった.1つは実際のShellに実装されているもので低対話型Honeypotに実装されていないコマンドの実装すること.もう一つは低対話型Honeypotに特有の異常な挙動を修正を行うことである.

\section{本研究の評価}
\label{intr:eval}
提案手法の実装で拡張した低対話型Honeypotと,素のHoneypotと,高対話型Honeypotを設置し,それぞれ侵入者が実行したコマンドのログを収集した.収集したコマンドのログはコマンド1つ1つごとに自然言語処理で意味解析をし,コマンドの意味をベクトル空間上に表現した.そして,拡張した低対話型Honeypotの侵入ログが素のHoneypotと比較して,高対話型Honeypotの侵入ログにどれほど次元空間上で近似したのかを評価した.

%\section{デジタルファブリケーションの発達}
%\label{intr:background}

%本節では,デジタルファブリケーションと呼ばれるものづくりの発達と,それに伴うパーソナルファブリケーションについて説明する.

%\section{デジタルファブリケーション}
%\label{intr:digitalfabrication}

%デジタルファブリケーションとは,3Dプリンタやレーザカッタといったコンピュータに接続されたデジタル工作機器を用いて3Dモデルを実際に造形物として成形する技術のことである.近年,コンピュータの普及とともに,デジタル工作機器は安価かつ小型になりつつある.また3Dモデリングソフトウェアもオープンソースのものが現れるなど,個人であっても高精度で3Dモデルを出力できる機器を入手,製造が行える環境ができつつある.


%\subsection{3Dプリントの普及と活用}
%\label{intr:3dprintrise}
%2000年代後半以降,技術の発達や低コスト化により,3Dプリンタが急速に普及している。3Dプリンタは樹脂素材などを加工し,設計図である3Dモデル同様の立体物を造形するデジタル工作機器である.1980年代の開発当初3Dプリンタは工業製品の試作のために製造業の中で主に使われていた.2005年にアメリカの3D Systems社が保持していた光造形法をはじめとする多くの造形手法が特許失効したことや,3Dプリンタ自体の製造技術の発達などの理由で,安価なものが作られるようになった.


%また,インターネット上で,3Dプリンタに関連する情報や,実際にプリントを行うための設計図などが,数多くWebで共有されている\cite{Thingiverse}.そうした情報を入手することで,今まで専門的な機器であった3Dプリンタを個人でも扱える環境が整いつつある.


%3Dプリンタの安価化と情報共有の迅速化・簡易化の二つの要因により,専門家以外による3Dプリンタを用いたものづくりは急速に普及している.3Dプリンタの市場規模は2015年の段階で11億ドルに対し,2019年には26億ドル超になると予測されている\cite{3dprintermarket}.そうした中で,現在Fablab\cite{FabLab}と呼ばれるデジタル工作機器を扱うことができる施設が世界各地で設置されており,デジタルファブリケーションの普及に努める拠点となっている.日本でも2010年以降神奈川県鎌倉市や茨城県つくば市を始めとして各地に設置されている.Fablabでは個人がデジタル工作機器を持たずとも,デジタルファブリケーションを行うことができる.


%3Dプリントの普及に伴い,開発当初想定されていた試作以外にも様々な応用が考えられるようになった.応用例の一つとして義足が挙げられる\cite{3dprintProstheticleg}.義足は使用者によってそのサイズや形状が異なるため,一律に大量生産することはできない.そのため,既存の義足の製作は,熟練した専門家によってオーダーメイドで制作されており,製作自体や修正も簡単ではない.3Dプリントであれば,3Dモデルを個人の形に合わせて改変することが容易である.修正する場合も,3Dモデルの修正と再出力は簡単に行うことができる.また,福田\cite{3dperson}は自分の実物大の3Dモデルを用いた作品を製作し,慶應義塾大学湘南藤沢キャンパス内で展示を行った.出力の際に3Dモデルの解像度を調整することで,作品の閲覧者が制作物から3Dモデルの元となった人物を特定できることをなどを確認した.これは芸術分野においても3Dモデルの改変や再出力によって様々な可能性があることを示唆するものである.

%\subsection{パーソナルファブリケーションとオープンデザイン}
%\label{intr:personalfabrication}
%3Dプリンタの普及に伴い,個人で製造を行うパーソナルファブリケーションと呼ばれるものづくりも行なわれつつある\cite{mota2011rise}.インターネットを通じて入手した3Dモデルを自分の環境に合わせて改変しプリントを行うなど,3Dモデルの2次利用,3次利用も行なわれている.インターネット上で公開した3Dモデルが,他人によって改変され派生が生まれる中で,2次利用者が加えた改変が元の3Dモデルに取り入れられることもある.これは,ソフトウェア開発におけるオープンソースと似た構造である.これらの動きから,オープンソースの考え方などをデザインに適用する``オープンデザイン"という概念も提唱されている\cite{opendesign}.


%\section{本研究の着目する課題と目的}
%\label{intr:issue}

%現在のデジタルファブリケーションでは個人で製造において,製造責任の追及や知的財産権の保証のために,製造物の製造情報の管理が一つの課題となっている.ここで扱う製造情報としては,設計図情報である3Dモデルデータ,3Dデータの設計者,製造者,製造日時などが含まれる.例えば,製造物によって事故が起こった場合,その責任を誰に求めるかといった製造責任(Product Liability,PL)の追求を行う.その際設計図などから設計上の欠陥を追及する必要性がある.また,自分の設計物を知的財産として証明する際にも製造情報が保存されていることが必要である.製造責任の追及や知的財産権の保障を行う為には,データが誰にでも参照できる公開性,製造物からデータが追跡できる追跡可能性,保存されたデータが後日改ざんされておらず完全性を保っていることが必要である.そこで,3Dプリントの際にRFIDを製造物に埋め込み,データサーバに保存された製造情報と製造物を紐付ける試みが行われている\cite{3dprintwithrfid}.この手法では,追跡可能性は担保されるものの,製造物に紐づけられた製造情報の完全性が担保されていない.


%本研究では,パーソナルファブリケーションによる個人的な製造が行われる中で,製造責任の知的財産権の所在を明らかにするシステムを提案した.

%\section{本研究の仮説}
%\label{intr:hypothesis}

%Bitcoinの基幹技術として発明されたBlockchain技術は,P2Pネットワーク上でデータが検証されたことを合意し公開台帳を形成するシステムである.公開台帳上に記録されたデータは各参加ノードによって分散的に保持され,改ざんを相互に監視するため,正規の手続きを踏まなければデータの更新もできず,データが失われる可能性も極めて低い.

%そこで本研究では,3Dプリントにおける製造情報をBlockchain上に保存することで,~\ref{intr:issue}節で述べた条件を満たし,製造責任や知的財産権の所在を明らかにできるのではないかと考えた.


%\section{本研究の手法}
%\label{intr:method}
%本研究では,3Dプリンタを制御するシングルボードコンピュータをEthereumノードとすることでBlockchain上に製造情報を保存する実験システムを構築した.EthereumとはBlockchainを状態遷移を記録する公開台帳として用いるためのアプリケーション開発プラットフォームである.Blockchain上に製造情報を保存できていることを確認し,本システムのスケーラビリティ,情報の改ざん耐性を推定することで,要件を満たせることを確認した.


\section{本論文の構成}
\label{intr:kosei}
%本論文の構成を以下に示す.第2章では,本研究の要素技術となるShellとHoneypotと自然言語処理について整理する.第3章では本研究にあたっての事前実験の概要と結果を述べ,第4章では,本研究の問題の定義をする.第5章では,本提案手法のシステムについて解説する.第6章では,実装方法や実装例について述べ,第7章で提案システムの評価手法について説明する.第8章では先行研究や事例について紹介する.第9章では本研究のまとめと展望について言及する.
本論文における以降の構成は次の通りである.

~\ref{tech}章では,本研究の要素技術となるShellとHoneypotと自然言語処理について整理する.
~\ref{appr}章では,本研究における問題の定義と,解決するための要件,仮説について説明する.
~\ref{prex}章では,本研究にあたっての事前実験の概要と結果を述べる.
~\ref{meth}章では,本提案手法について解説する.
~\ref{impl}章では,Honeypotの拡張についての実装方法や実装例について述べる.
~\ref{eval}章では,求められた課題に対しての評価を行い,考察する.
~\ref{rela}章では,関連研究を紹介し,本研究との比較を行う.
~\ref{conc}章では,本研究のまとめと今後の課題,展望についてまとめる.


%%% Local Variables:
%%% mode: japanese-latex
%%% TeX-master: "../thesis"
%%% End:
