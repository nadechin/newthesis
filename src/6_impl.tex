\chapter{実装}
\label{impl}

本章では,低対話型Honeypotと高対話型Honeypotの設置環境についても示し,~\ref{meth:appr}節で述べた手法を用いて純正のHoneypotにどのようなコマンドを実装し,Honeypot特有の異常な挙動を修正したのかを説明する.

\section{実装環境}
本研究で実装するシステムを構成するためのハードウェアおよびソフトウェアについて説明する.表\ref{table:imple}に詳細なバージョンを示す.
\label{impl:env}

% -------------------
\vspace{3mm}
%\newlength{\myheight}
\setlength{\myheight}{10mm}
\begin{table}[h]
 \caption{実装環境}
 \label{table:imple}
 \centering
  \begin{tabular}{|c|c|c|}
   \hline
   ハードウェア/ソフトウェア & 実装環境 & Version(date)  \\
    \hline \hline
     \parbox[c][\myheight][c]{0cm}{} 純正のCowrie  & CentOS5 & 1.4.0  \\
     \hline
     \parbox[c][\myheight][c]{0cm}{} 修正済みのCowrie  & CentOS5 & 1.4.0(self made)  \\
     \hline
     \parbox[c][\myheight][c]{0cm}{} Honeywall  & CentOS5 & 1.4  \\
     \hline
  \end{tabular}
\end{table}
\vspace{7mm}
% --------------------

\subsection{純正のHoneypotで未実装のコマンド種類}
\label{impl:ImplBusyBox}
本研究において純正のHoneypotはCowrie\cite{cowrie}を使用し,実際のShellには実装されているが,純正のHoneypotで未実装のコマンドについてはBusyBox\cite{busybox}に含まれるコマンドの実装を行なった.~\ref{tech:BusyBox}や~\ref{Cowrie}で紹介した通り,BusyBoxに含まれるコマンドの種類が219ある中で,Cowrieの実装コマンド数は92しか存在しない.この差分をPythonで実装する. \\
またBusyBoxに含まれるコマンドとCowrieの実装コマンド,Cowrieに未実装のコマンドの一覧は付録~\ref{appendix}の表\ref{table:command}に示しておく. \\

\subsection{純正のHoneypotで未実装のコマンドの実装}
~\ref{impl:ImplBusyBox}で示したCowrieに未実装のコマンドについての一部の実装を示す.残りの実装は~\ref{appendix:implofcommand}の表に示す.


%\vspace{3mm}
%\setlength{\myheight}{10mm}
%\begin{table}[h]
% \caption{実装コマンド一覧}
% \label{table:command}
% \centering
%  \begin{tabular}{|c|c|}
%   \hline
%   Busyboxに含まれるコマンド & Cowrieの実装コマンド  \\
%    \hline \hline
%  \end{tabular}
%\end{table}
%\vspace{7mm}
% --------------------

%%% Local Variables:
%%% mode: japanese-latex
%%% TeX-master: "../bthesis"
%%% End:
