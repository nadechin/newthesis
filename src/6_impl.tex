\chapter{実装}
\label{impl}

本章では,低対話型Honeypotと高対話型Honeypotの設置環境についても示し,~\ref{meth:appr}節で述べた手法を用いて純正のHoneypotにどのようなコマンドを実装し,Honeypot特有の異常な挙動を修正したのかを説明する.

\section{実装環境}
本研究で実装するシステムを構成するためのハードウェアおよびソフトウェアについて説明する.表\ref{table:imple}に詳細なバージョンを示す.
\label{impl:env}

% -------------------
\vspace{3mm}
%\newlength{\myheight}
\setlength{\myheight}{10mm}
\begin{table}[h]
 \caption{実装環境}
 \label{table:imple}
 \centering
  \begin{tabular}{|c|c|c|}
   \hline
   ハードウェア/ソフトウェア & 実装環境 & Version(date)  \\
    \hline \hline
     \parbox[c][\myheight][c]{0cm}{} 純正のCowrie  & CentOS5 on SAKURA Internet VPS & 1.4.0  \\
     \hline
     \parbox[c][\myheight][c]{0cm}{} 修正済みのCowrie  & CentOS5 on SAKURA Internet VPS & 1.4.0(self made)  \\
     \hline
     \parbox[c][\myheight][c]{0cm}{} Honeywall  & CentOS5 on SAKURA Internet VPS & 1.4  \\
     \hline
  \end{tabular}
\end{table}
\vspace{7mm}
% --------------------

\subsection{純正のHoneypotで未実装のコマンド種類}
\label{impl:ImplBusyBox}
本研究において純正のHoneypotはCowrie\cite{cowrie}を使用し,実際のShellには実装されているが,純正のHoneypotで未実装のコマンドについてはBusyBox\cite{busybox}に含まれるコマンドの実装を行なった.~\ref{tech:BusyBox}や~\ref{Cowrie}で紹介した通り,BusyBoxに含まれるコマンドの種類が219ある中で,Cowrieの実装コマンド数は92しか存在しない.この差分をPythonで実装する. \\
またBusyBoxに含まれるコマンドとCowrieの実装コマンド,Cowrieに未実装のコマンドの一覧は付録~\ref{appd}の表\ref{table:command}に示しておく. \\

\subsection{純正のHoneypotで未実装のコマンドの実装}
~\ref{impl:ImplBusyBox}で示したCowrieに未実装のコマンドについての一部の実装を示す.他の実装は~\ref{appd:implofcommand}の表に示す.
Cowrieに実装されているコマンドはcowrie/src/cowrie/commands/に格納されており,ここに追加コマンドを実装する.\\

以下の\ref{lst:dmidecode}にcowrie/src/cowrie/commands/base.pyにdmidecodeコマンドを追加実装したものを示す.\cite{dmidecode}

\begin{mylisting}[label={lst:dmidecode},language=sh,caption=dmidecode]
class command_dmidecode(HoneyPotCommand):
    """
    """
    def call(self):
        """
        """
        self.write(b"""# dmidecode 3.1
Getting SMBIOS data from sysfs.
SMBIOS 3.0.0 present.
Table at 0xDDBA4000.

(snip)

Handle 0x0055, DMI type 13, 22 bytes
BIOS Language Information
    Language Description Format: Long
    Installable Languages: 1
        en|US|iso8859-1
    Currently Installed Language: en|US|iso8859-1

Handle 0x0056, DMI type 127, 4 bytes
End Of Table\n""")
commands['dmidecode'] = command_dmidecode
\end{mylisting}

追加実装以外ではファイルを追加することで,コマンドを追加したものを以下の\ref{lst:diff}に示す.\cite{diffcommand}

\begin{mylisting}[label={lst:diff},language=sh,caption=diffコマンド]
# coding: utf-8
import sys

class Diff:
  def __init__(self, a, b):
    self.a = a
    self.b = b

  def solve(self):
    self.result = None
    self.createTable()
    self.selectMatches()
    self.connectPath()
    self.genResult()
    return self.result

  def createTable(self):
    self.table = [[x == y for y in self.b] for x in self.a]

  def selectMatches(self):
    self.matches = [ (i, j)
      for i in range(len(self.table))
      for j in range(len(self.table[i]))
      if self.table[i][j]]

    self.max_path = []
    self.path = []

    self.search((-1, -1)) # 左上から探索
    # print self.max_path

  def search(self, pos):
    def isReachable(pos, match):
      return match[0] > pos[0] and match[1] > pos[1]

    if pos != (-1, -1): self.path.append(pos)
    is_term = True
    for match in self.matches:
      if isReachable(pos, match):
        is_term = False
        next_pos = match
        self.search(next_pos)

    if is_term: # 終端の場合
      if len(self.path) > len(self.max_path):
        self.max_path = list(self.path) # 最良経路の更新
    if pos != (-1, -1): self.path.pop()

  def connectPath(self):
    # self.path = [(0, 0)]
    self.path = []
    prev = (0, 0)
    for pos in self.max_path:
      p = prev
      while p[0] < pos[0]:
        p = (p[0] + 1, p[1])
        self.path.append(p)
      while p[1] < pos[1]:
        p = (p[0], p[1] + 1)
        self.path.append(p)
      p = (p[0] + 1, p[1] + 1)
      self.path.append(p)
      prev = p
    p = prev
    while p[0] < len(self.table):
      p = (p[0] + 1, p[1])
      self.path.append(p)
    while p[1] < len(self.table[0]):
      p = (p[0], p[1] + 1)
      self.path.append(p)
    # print self.path

  def genResult(self):
    prev = (0, 0)
    i_a = 0
    i_b = 0
    self.result = []
    for pos in self.path:
      if pos == (0, 0): continue
      if pos[1] == prev[1]: # 縦への移動 : a の -
        self.result.append((self.a[i_a], '-'))
        i_a += 1
      elif pos[0] == prev[0]: # 横への移動 : b の +
        self.result.append((self.b[i_b], '+'))
        i_b += 1
      else: # 斜めへの移動
        self.result.append((self.a[i_a], ' '))
        i_a += 1
        i_b += 1
      prev = pos
      # print self.result

  def printResult(self):
    for line, sign in self.result:
      print sign, line,

def diff(filename_a, filename_b):
  a = []
  with open(filename_a) as f:
    for line in f:
      a.append(line)

  b = []
  with open(filename_b) as f:
    for line in f:
      b.append(line)

  # a = ['a', 'b', 'c', 'd', 'e', 'f', 'g']
  # b = ['a', 'b', 'x', 'c', 'y', 'e', 'g']
  solver = Diff(a, b)
  result = solver.solve()
  solver.printResult()

if __name__ == "__main__":
  if len(sys.argv) != 3:
    print "Usage: python %s fileA fileB" % sys.argv[0]
    quit()
  diff(sys.argv[1], sys.argv[2])
\end{mylisting}

さらに,コマンド実行時に出力結果をprint関数で出力するのみのコマンド追加したものの例を以下の\ref{lst:mount}示す.本例ではmountコマンドを以下のように実装した.

\begin{mylisting}[label={lst:mount},language=sh,caption=mountコマンド]
# coding: utf-8

string = '''sysfs on /sys type sysfs (rw,nosuid,nodev,noexec,relatime)
proc on /proc type proc (rw,nosuid,nodev,noexec,relatime)
udev on /dev type devtmpfs (rw,nosuid,relatime,size=16459820k,nr_inodes=4114955,mode=755)
devpts on /dev/pts type devpts (rw,nosuid,noexec,relatime,gid=5,mode=620,ptmxmode=000)
tmpfs on /run type tmpfs (rw,nosuid,noexec,relatime,size=3294084k,mode=755)
/dev/sda1 on / type ext4 (rw,relatime,discard,data=ordered)
securityfs on /sys/kernel/security type securityfs (rw,nosuid,nodev,noexec,relatime)
tmpfs on /dev/shm type tmpfs (rw,nosuid,nodev)
tmpfs on /run/lock type tmpfs (rw,nosuid,nodev,noexec,relatime,size=5120k)
tmpfs on /sys/fs/cgroup type tmpfs (ro,nosuid,nodev,noexec,mode=755)
cgroup on /sys/fs/cgroup/systemd type cgroup (rw,nosuid,nodev,noexec,relatime,xattr,release_agent=/lib/systemd/systemd-cgroups-agent,name=systemd)
pstore on /sys/fs/pstore type pstore (rw,nosuid,nodev,noexec,relatime)
cgroup on /sys/fs/cgroup/net_cls,net_prio type cgroup (rw,nosuid,nodev,noexec,relatime,net_cls,net_prio)
cgroup on /sys/fs/cgroup/freezer type cgroup (rw,nosuid,nodev,noexec,relatime,freezer)
cgroup on /sys/fs/cgroup/pids type cgroup (rw,nosuid,nodev,noexec,relatime,pids)
cgroup on /sys/fs/cgroup/cpuset type cgroup (rw,nosuid,nodev,noexec,relatime,cpuset)
cgroup on /sys/fs/cgroup/cpu,cpuacct type cgroup (rw,nosuid,nodev,noexec,relatime,cpu,cpuacct)
cgroup on /sys/fs/cgroup/memory type cgroup (rw,nosuid,nodev,noexec,relatime,memory)
cgroup on /sys/fs/cgroup/devices type cgroup (rw,nosuid,nodev,noexec,relatime,devices)
cgroup on /sys/fs/cgroup/hugetlb type cgroup (rw,nosuid,nodev,noexec,relatime,hugetlb)
cgroup on /sys/fs/cgroup/rdma type cgroup (rw,nosuid,nodev,noexec,relatime,rdma)
cgroup on /sys/fs/cgroup/blkio type cgroup (rw,nosuid,nodev,noexec,relatime,blkio)
cgroup on /sys/fs/cgroup/perf_event type cgroup (rw,nosuid,nodev,noexec,relatime,perf_event)
systemd-1 on /proc/sys/fs/binfmt_misc type autofs (rw,relatime,fd=26,pgrp=1,timeout=0,minproto=5,maxproto=5,direct,pipe_ino=11662)
debugfs on /sys/kernel/debug type debugfs (rw,relatime)
mqueue on /dev/mqueue type mqueue (rw,relatime)
hugetlbfs on /dev/hugepages type hugetlbfs (rw,relatime,pagesize=2M)
configfs on /sys/kernel/config type configfs (rw,relatime)
fusectl on /sys/fs/fuse/connections type fusectl (rw,relatime)
/dev/sdb1 on /mnt type ext4 (rw,relatime,data=ordered)
lxcfs on /var/lib/lxcfs type fuse.lxcfs (rw,nosuid,nodev,relatime,user_id=0,group_id=0,allow_other)
tmpfs on /run/user/1000 type tmpfs (rw,nosuid,nodev,relatime,size=3294084k,mode=700,uid=1000,gid=1000)'''

print(string)
\end{mylisting}

残りのコマンドについては~\ref{appd:implofcommand}に記す.

%\vspace{3mm}
%\setlength{\myheight}{10mm}
%\begin{table}[h]
% \caption{実装コマンド一覧}
% \label{table:command}
% \centering
%  \begin{tabular}{|c|c|}
%   \hline
%   Busyboxに含まれるコマンド & Cowrieの実装コマンド  \\
%    \hline \hline
%  \end{tabular}
%\end{table}
%\vspace{7mm}
% --------------------

%%% Local Variables:
%%% mode: japanese-latex
%%% TeX-master: "../bthesis"
%%% End:
