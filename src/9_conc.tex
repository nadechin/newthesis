\chapter{結論}
\label{conc}

本章では,本研究のまとめと今後の課題を示す.

\section{本研究のまとめ}

本研究では,侵入者からHoneypotであることの検知を回避するために,低対話型Honeypotの実行コマンドの挙動を本物のShellの実行コマンドの挙動に近づけることを提案した.侵入者からHoneypotであることの検知を回避するためには,本物のShellに実装されているコマンドの実装と,低対話型Honeypot特有の異常な挙動をするコマンドの修正を行う必要がある.そこで,本物のShellに実装されているコマンドを低対話型Honeypotに全て実装し,低対話型Honeypot特有の異常な挙動をするコマンドの修正を行った.低対話型Honeypotの実行コマンドの挙動を本物のShellの実行コマンドも挙動に近づけたかを検証するために,追加実装を施していない低対話型Honeypotとコマンド拡張を行なった低対話型Honeypot,高対話型Honeypotを設置してコマンドログを収集し,一般のUNIXユーザーの実行コマンドログと比較した.コマンドログの自然言語処理による意味解析を行ない,個々のコマンドの意味を多次元のベクトル空間上で表現することで,コマンド拡張を行なった低対話型Honeypotのコマンドログが素の低対話型Honeypotと比較して,一般的なUNIXユーザーの実行するコマンドログから離れたことを明らかにした.このことから,素のHoneypotにコマンド拡張を行い,低対話型Honeypotの実行コマンドの挙動を本物のShellの実行コマンドの挙動に近づけることで,素の低対話型Honeypotと修正済のHoneypot,高対話型Honeypotの攻撃ログを,SCDVを用いた文章ベクトル空間上で比較した結果,修正済のHoneypotのコマンドログは一般ユーザーのコマンドログを始点とすると,素のHoneypotのコマンドログのベクトル方向よりも正の向きに遠くに位置することが分かった.また,素のHoneypotのコマンドログと比較して,修正済のHoneypotは一般のUnixユーザーのコマンドログにとっての異常なコマンドログを収集でき,修正済の低対話型Honeypotで取れたコマンドログが様々な攻撃パターンと異常な挙動を収集できることを示した.

\section{本研究の課題と展望}
本節では,提案手法の課題とその展望を述べる.SSHの低対話型Honeypotに実装するコマンドについて,ディストリビューションごとに実装コマンドが異なるが,今回はBusyBoxをそのままPythonで実装した.したがって,低対話型Honeypotがエミュレーションしているディストリビューションで実装されているコマンドは必要条件しか満たしていない.また,~\ref{appr:problemofSshLowHoneypot}で述べた通り,
低対話型Honeypotには本研究と着目した問題以外にも,SSHでのセッション確立におけるレイテンシの問題や,HoneypotのUsernameの問題も存在する.しかし本研究ではこれらを扱わず,本物のShellの挙動を完全にエミュレートできていないため,侵入者からHoneypotであると検知されてしまう余地がある.

\subsection{文章ベクトルの評価指標}

~\ref{eval:CommandVector}節において,コマンドログをSCDVで文章ベクトル化することで評価したが,その評価指標として,Accuracy,precision,recall,f1-scoreを算出した.また,この評価指標の内,Accuracyによって文章の近さを測定したが,これが一番評価指標として適切か否かについての議論が本研究で行わなかった為,評価の再検討をする余地がある.

\subsection{高対話型Honeypotのログ収集の不足}
本研究においてhoneywallを用いた高対話型Honeypotのコマンドログの収集を行なったが,収集ログ数が極端に少なく,またその原因をつかむことができなかった.しかし,収集ログの数が現行の低対話型HoneypotのCowrieが圧倒的に収集することができたことを示すことができた.高対話型Honeypotの設置と多数の収集ログ数の取得によって,本物のOSと差のないコマンドログを収集することができる為,コマンド実行における”攻撃性”をベクトル空間上で示すことができる.これによって低対話型Honeypotにおける攻撃性のあるコマンドログが収集できたかを評価することができる.

\subsection{ネットワークセグメント毎のログ収集環境の違い}
2018年度の慶應義塾大学が主催するORFにおいて,ORF-NOCとして従事し,その過程でrgのネットワーク内に低対話型Honeypotを設置した.その結果,本研究における一日あたりのコマンドログの収集数よりも約1.5倍もの多くのコマンドログの収集数を取得することができた.したがって,同一のHoneypotでも配置する環境によって大きな差が出ることがわかり,設置する環境について考慮する余地がある.

\subsection{ディストリビューションごとの実装コマンド}
ディストリビューションごとの実装コマンドについてはバージョンの問題にも依存する.しかし,OSのリリースにはLTS(Long Term Support)があり,広く使われているOSでの実装コマンドの検証は十分に可能である.

\subsection{様々なHoneypotのコマンドログの精度評価への応用}
現在様々な種類のSSHの低対話型Honeypotが普及しているおり,それらのHoneypotごとに実装コマンドが異なる.そのため,Honeypotの種類ごとに攻撃ログを収集し,本研究の評価手法を用いることで,自然言語処理による意味解析において使用したHoneypotで取れたコマンドログが,高対話型Honeypotのコマンドログにどれほど空間的距離が近いのかを検証することができる.

\subsection{コマンド系ごとの評価への応用}
SSHの低対話型Honeypotを設置し,自然言語処理による意味解析を行なったコマンドログにおいて,個々のコマンドごとに持つ意味が異なる,あるコマンドが使用できないような実装を施すと高対話型Honeypotのコマンドログにどれほど空間的距離が遠くなるかを検証できる.
%%% Local Variables:
%%% mode: japanese-latex
%%% TeX-master: "../thesis"
%%% End:
