\chapter{結論}
\label{conc}

本章では,本研究のまとめと今後の課題を示す.

\section{本研究のまとめ}

本研究では,低対話型Honeypotにおいて,高対話型Honeypotの侵入ログに近似するためには,侵入者からHoneypotであることの検知を回避する必要がある.そのために,低対話型Honeypotの実行コマンドの挙動を本物のShellの実行コマンドの挙動に近づけることを提案した.侵入者からHoneypotであることの検知を回避するためには,本物のShellに実装されているコマンドの実装と,低対話型Honeypot特有の異常な挙動をするコマンドの修正を行う必要がある.そこで,本物のShellに実装されているコマンドを低対話型Honeypotに全て実装し,低対話型Honeypot特有の異常な挙動をするコマンドの修正を行った.低対話型Honeypotの実行コマンドの挙動を本物のShellの実行コマンドも挙動に近づけたかを検証するために,追加実装を施していない低対話型Honeypotとコマンド拡張を行なった低対話型Honeypot,高対話型Honeypotを設置してコマンドログを収集し,比較した.コマンドログの自然言語処理による意味解析を行ない,個々のコマンドの意味を多次元のベクトル空間上で表現することで,コマンド拡張を行なった低対話型Honeypotのコマンドログが高対話型Honeypotのコマンドログの空間的距離が近いということが明らかにした.このことから,素のHoneypotにコマンド拡張を行い,低対話型Honeypotの実行コマンドの挙動を本物のShellの実行コマンドの挙動に近づけることで,拡張した低対話型Honeypotで取れたコマンドログが高対話型Honeypotのコマンドログに近似することを確認した.

\section{本研究の課題と展望}
本節では,提案手法の課題とその展望を述べる.SSHの低対話型Honeypotに実装するコマンドについて,ディストリビューションごとに実装コマンドが異なるが,今回はBusyBoxをそのままPythonで実装した.したがって,低対話型Honeypotがエミュレーションしているディストリビューションで実装されているコマンドは必要条件しか満たしていない.また,~\ref{appr:problemofSshLowHoneypot}で述べた通り,
低対話型Honeypotには本研究と着目した問題以外にも,SSHでのセッション確立におけるレイテンシの問題や,HoneypotのUsernameの問題も存在する.しかし本研究ではこれらを検証していないため,本物のShellの挙動を完全にエミュレートできたわけではないため,侵入者からHoneypotであることを検知されてしまう可能性はまだ残されている.

\subsection{ディストリビューションごとの実装コマンド}
ディストリビューションごとの実装コマンドについてはバージョンの問題にも依存する.しかし,OSのリリースにはLTS(Long Term Support)があり,広く使われているOSでの実装コマンドの検証は十分に可能である.

\subsection{様々なHoneypotのコマンドログの精度評価への応用}
現在様々な種類のSSHの低対話型Honeypotが普及しているおり,それらのHoneypotごとに実装コマンドが異なる.そのため,Honeypotの種類ごとに攻撃ログを収集し,本研究の評価手法を用いることで,自然言語処理による意味解析において使用したHoneypotで取れたコマンドログが,高対話型Honeypotのコマンドログにどれほど空間的距離が近いのかを検証することができる.

\subsection{コマンド系ごとの評価への応用}
SSHの低対話型Honeypotを設置し,自然言語処理による意味解析を行なったコマンドログにおいて,個々のコマンドごとに持つ意味が異なる,あるコマンドが使用できないような実装を施すと高対話型Honeypotのコマンドログにどれほど空間的距離が遠くなるかを検証できる.
%%% Local Variables:
%%% mode: japanese-latex
%%% TeX-master: "../thesis"
%%% End:
